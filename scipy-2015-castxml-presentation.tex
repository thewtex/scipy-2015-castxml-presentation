%% LyX 2.0.6 created this file.  For more info, see http://www.lyx.org/.
%% Do not edit unless you really know what you are doing.
\documentclass[english]{beamer}
\usepackage{mathptmx}
\usepackage[T1]{fontenc}
\usepackage[latin9]{inputenc}
\usepackage{amsmath}
\usepackage{amssymb}

\makeatletter
%%%%%%%%%%%%%%%%%%%%%%%%%%%%%% Textclass specific LaTeX commands.
 % this default might be overridden by plain title style
 \newcommand\makebeamertitle{\frame{\maketitle}}%
 \AtBeginDocument{
   \let\origtableofcontents=\tableofcontents
   \def\tableofcontents{\@ifnextchar[{\origtableofcontents}{\gobbletableofcontents}}
   \def\gobbletableofcontents#1{\origtableofcontents}
 }
 % plain title style, override default
 \renewcommand\makebeamertitle{\frame[plain]{\maketitle}}%
 \long\def\lyxframe#1{\@lyxframe#1\@lyxframestop}%
 \def\@lyxframe{\@ifnextchar<{\@@lyxframe}{\@@lyxframe<*>}}%
 \def\@@lyxframe<#1>{\@ifnextchar[{\@@@lyxframe<#1>}{\@@@lyxframe<#1>[]}}
 \def\@@@lyxframe<#1>[{\@ifnextchar<{\@@@@@lyxframe<#1>[}{\@@@@lyxframe<#1>[<*>][}}
 \def\@@@@@lyxframe<#1>[#2]{\@ifnextchar[{\@@@@lyxframe<#1>[#2]}{\@@@@lyxframe<#1>[#2][]}}
 \long\def\@@@@lyxframe<#1>[#2][#3]#4\@lyxframestop#5\lyxframeend{%
   \frame<#1>[#2][#3]{\frametitle{#4}#5}}
 \newenvironment{topcolumns}{\begin{columns}[t]}{\end{columns}}
 \def\lyxframeend{} % In case there is a superfluous frame end

%%%%%%%%%%%%%%%%%%%%%%%%%%%%%% User specified LaTeX commands.
\usetheme{Kitware}

\makeatother

\usepackage{babel}
\begin{document}





\title[CastXML]{Wrapping C and C++ Libraries with CastXML}


%\subtitle{Include Only If Paper Has a Subtitle}


\author[King, Brad]{Brad~King\inst{1}, Bill Hoffman\inst{1}, Matt
McCormick\inst{1}, and Michka Popoff}


\institute[Kitware, Inc]{\inst{1}Kitware, Inc.}


\date[2015-07-10]{SciPy, 2015}

\makebeamertitle


\AtBeginSubsection[]{

  \frame<beamer>{ 

    \frametitle{Outline}   

    \tableofcontents[currentsection,currentsubsection] 

  }

}




%\beamerdefaultoverlayspecification{<+->}


\lyxframeend{}\lyxframe{Outline}

\tableofcontents{}




\lyxframeend{}\section{Motivation}


\lyxframeend{}\subsection[Basic Problem]{The Basic Problem That We Studied}


\lyxframeend{}\lyxframe{Make Titles Informative. Use Uppercase Letters.}


\framesubtitle{Frame subtitles are optional. Use upper- or lowercase letters.}
\begin{itemize}
\item Use Itemize a lot.


\pause{}

\item Use very short sentences or short phrases.


\pause{}

\item These overlays are created using the Pause style.
\end{itemize}

\lyxframeend{}\lyxframe{Make Titles Informative. }
\begin{itemize}
\item <1->You can also use overlay specifications to create overlays.
\item <3->This allows you to present things in any order.
\item <2->This is shown second.
\end{itemize}

\lyxframeend{}\lyxframe{Make Titles Informative.}
\begin{block}
<1->{}
\begin{itemize}
\item Untitled block.
\item Shown on all slides.
\end{itemize}
\end{block}
\begin{exampleblock}
<2->{Some Example Block Title}
\begin{itemize}
\item $e^{i\pi}=-1$.
\item $e^{i\pi/2}=i$.
\end{itemize}
\end{exampleblock}

\lyxframeend{}\subsection{Previous Work}


\lyxframeend{}\lyxframe{Make Titles Informative. }
\begin{example}%{}
<1->On first slide. 
\end{example}%{}

\begin{example}%{}
<2->On second slide.
\end{example}%{}

\lyxframeend{}\section{Our Results/Contribution}


\lyxframeend{}\subsection{Main Results}


\lyxframeend{}\lyxframe{Make Titles Informative. }
\begin{theorem}%{}
On first slide.
\end{theorem}%{}

\pause{}
\begin{corollary}%{}
On second slide.
\end{corollary}%{}

\lyxframeend{}\lyxframe{Make Titles Informative. }
\begin{topcolumns}%{}


\column{5cm}
\begin{theorem}%{}
<1->In left column.
\end{theorem}%{}

\column{5cm}
\begin{corollary}%{}
<2->In right column.\\
New line
\end{corollary}%{}
\end{topcolumns}%{}

\lyxframeend{}\subsection{Basic Ideas for Proofs/Implementations}


\lyxframeend{}\section*{Summary}


\lyxframeend{}\lyxframe{Summary}
\begin{itemize}
\item The \textcolor{red}{first main message} of your talk in one or two
lines.
\item The \textcolor{red}{second main message} of your talk in one or two
lines.
\item Perhaps a \textcolor{red}{third message}, but not more than that.
\end{itemize}


\vskip0pt plus.5fill
\begin{itemize}
\item Outlook

\begin{itemize}
\item What we have not done yet.
\item Even more stuff.
\end{itemize}
\end{itemize}

\lyxframeend{}

\appendix

\lyxframeend{}\section*{Appendix}


\lyxframeend{}\subsection*{For Further Reading}


\lyxframeend{}\lyxframe{[allowframebreaks]For Further Reading}

\beamertemplatebookbibitems
\begin{thebibliography}{1}
\bibitem{Author1990}A. Author. \newblock\emph{Handbook of Everything}.\newblock
Some Press, 1990.\beamertemplatearticlebibitems

\bibitem{Someone2002}S. Someone.\newblock On this and that\emph{.}
\newblock\emph{Journal on This and That}. 2(1):50--100, 2000.

\end{thebibliography}

\lyxframeend{}
\end{document}
